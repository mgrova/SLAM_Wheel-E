\section{Introducción al SLAM\textit{(Simultaneous Localization and Mapping)}}
¿Qué cojones es el SLAM?

\subsection{Clasificación técnicas de SLAM}
La forma en que se toma información de las imagenes que recibe y de su entorno y cómo la procesa permite distinguir las siguientes tipos de sistemas
SLAM visual:
\begin{itemize}
    \item \textit{Denso vs Disperso} \\
    En función de la cantidad de datos tomada de cada imagen, se realizará ésta clasificación. Las técnicas de SLAM disperso sólo emplean una pequeña
    región de pixeles, cómo pueden ser puntos característicos. Sin embargo, las técnicas densas emplean la mayoría o todos los píxeles de cada frame
    que reciben. \\
    Debido a que la cantidad y tipo de información que toman son diferentes, el tipo de mapa que se obtendrá también lo es. Con las técnicas de SLAM
    disperso se obtendrá una nube de puntos que será una representación de los puntos característicos de las escena y se usará sobre todo para hacer
    un tracking de la pose de la cámara. Por otro lado, un mapa dentro tendrá muchos más detalles y, requerirá de una mucho más elevada carga computacional.

 \item \textit{Directo vs Indirecto} \\
En función de cómo las técnicas de SLAM empleen y traten la información que reciben, se podrá clasificar en SLAM directo o indirecto.\\
El SLAM infirecto, se basa en extraer primero las features de la imagen, puntos característicos, para posteriormente emplearlos para localizarse y contruir
el mapa. Para extraer estos puntos caracteristicos existen muchos descriptores: ORB, SIFT, FAST, etc. \\
En contraste, el SLAM Directo,emplean directamente la intensidad de los pixeles, de tal modo que se obtengan features intermedios. Estos metodos tratan de 
obtener la profundidad y estructura del entorno a partir de una optimización del mapa. Los procesos de extracción de puntos característicos son mucho más
pesados computacionalmente si se trabaja al mismo rate que un slam indirecto.\\
Por último, destacar, que los metodos indirectos de SLAM son más tolerantes a los cambios de iluminación del entorno.\\

A continuación, se monstarán una comparativa de una serie de técnicas de SLAM:
\begin{figure}[h!]
    \centering
    \includegraphics[width=.6\textwidth]{images/comp_slam}
    \caption{Comparativa de técnicas SLAM}
\end{figure}

\end{itemize}
En éste proyecto, se probará ORB-SLAM2 y RTAB-Map, los cuales abordan el problema del SLAM desde dos puntos de vista completamente distinto, cómo se
verá a continuación.
\subsection{\textit{RTAB-Map SLAM}}
RTAB-Map \textit{(Real-Time Appearance-Based Mapping)} es una técnica de Graph-SLAM\footnote{http://robots.stanford.edu/papers/thrun.graphslam.pdf} basada en la detección de bucles cerrados incrementales. Es totalmente funcional 
con sensores RGB-D, Stereo y LIDAR. \\
El detector de bucles cerrados se basará en la comparativa de cuán semenjantes son la imagenes en una localización y la previa. Cuándo una hipótesis 
de bucle cerrado es aceptada, se añade una nueva restricción al \textit{graph} del mapa y, tras ello, el optimizador minimiza el error del mapa. \\

\begin{figure}[h!]
    \centering
    \includegraphics[width=.7\textwidth]{images/rtabmap_scheme}
    \caption{Esquema del Back-End y Front-End de RTAB-Map}
\end{figure}


\subsubsection{Fundamento teórico de la técnica}
Este algoritmo plantea una estrategia de particionado de memoria que pretende asemejarse al funcionamiento de la memoria humana, 
dónde ésta se estructura en: \\
Memoria de trabajo del robot \textit{(Working Memory)}, la memoria a largo plazo \textit{(Long Term Memory)}, 
la memoria a corto plazo \textit{(Short-term memory)} y la memoria sensorial \textit{(Sensory Memory)}. \\
De ese modo, se mantendrán en la memoria de trabajo del robot aquellas localizaciones que se han visitado recientemente 
y con más frecuencia, mientras que el resto pasarán a la memoria de largo plazo. \\

\begin{figure}[h!]
    \centering
    \includegraphics[width=.4\textwidth]{images/rtabmap_memory}
    \caption{Estructura de memoria de la técnica RTAB-Map}
\end{figure}


Se partirá de la premisa de que aquellas localizaciones que son visitadas de forma más frecuente son más propensas a 
crear bucles cerrados. Por ello, el número de veces que una localización sea visitada será empleado como peso, de 
esta forma serán transferidas desde la memoria de trabajo a la memoria de largo plazo aquellas observaciones que 
tengan mayor peso.\\
La memoria a corto plazo, \textit{STM}, tiene como misión buscar las similitudes que existan entre dos imágenes
consecutivas, mientras que la memoria de trabajo, \textit{WM}, es la encargada de detectar los bucles cerrados entre las
localizaciones en el espacio. El número de localizaciones almacenadas en la memoria del trabajo del robot es
limitado. El tamaño de la memoria a corto plazo, \textit{STM}, está basado en la velocidad del robot y en la frecuencia de
adquisición de las localizaciones. \\

% NO SE SI METER ESTA PARTE
\begin{comment}
Es importante destacar que todas las observaciones almacenadas en la memoria a largo plazo, \textit{STM}, del robot no se
usan para detectar bucles cerrados. No obstante, es importante elegir cuidadosamente las localizaciones que se
almacenan en esta memoria. Almacenar las observaciones según la técnica FIFO (First In First Out), sería un
error debido a que, como el algoritmo establece un número máximo de localizaciones que se pueden
almacenar mientras se está explorando un entorno, podríamos alcanzar el superar el umbral de tiempo
establecido sin llegar a cerrar el bucle haciendo que las localizaciones más antiguas nunca consigan asociarse. \\
Como alternativa, el orden de almacenamiento de las observaciones se elige de forma aleatoria, aunque es
preferible mantener en la memoria de trabajo aquellas localizaciones que son más susceptibles de ser
observadas.
\end{comment}
\subsubsection{Análisis de resultados obtenidos}

\newpage
\subsection{\textit{ORB-SLAM 2}}
ORB-SLAM2 es una técnica de SLAM en tiempo real para cámaras Monocular, Stereo y RGB-D que se engloba dentro de las técnicas de 
\textit{Sparse-Slam}. Se computa la trayectoria y hace una reconstrucción dispersa 3D del entorno. Al igual que todas las técnicas 
de SLAM, se basa en la detección de bucles cerrados y  se relocaliza en tiempo real. \\
Se basa en la detección de \textit{keyframes} que emplea para hacer un tracking de los mismos y a partir de ello crear el 
mapa local que, posteriormente, se optimizará junto al mapa global. \\
Unos de los puntos destacables de ésta técnica de SLAM son los siguientes:
\begin{itemize}
    \item Emplea \textit{grafo de covisibilidad}. Tanto el seguimiento como el mapping se focalizan en el área covisible,
    independientemente del tamaño del mapa completo, consiguiendo así explorar entornos amplios sin
    aumentar el tiempo y la carga de computación.

    \item La estrategia para detectar los bucles cerrados de visión en tiempo real se basa en la optimización de
   un de grafo denominado \textit{Essencial Graph}, lo cuál se desarrollará mas adelante.
\end{itemize}

\subsubsection{Fundamento teórico de la técnica}
A continuación, se tratará un poco el funcionamiento interno del algoritmo, el cuál presenta el siguiente esquema:
\begin{figure}[h!]
    \centering
    \includegraphics[width=1\textwidth]{images/orb_scheme}
    \caption{Estructura interna de ORB-SLAM2}
\end{figure}

Se observa cómo existen 6 grandes módulos, los cuales de lanzarán en 3 hilos paralelos. \\
En primer lugar, se tendrá el preprocesamiento de la imagen, de la cuál se obtendrán  ORB features 
\textit{(Oriented Fast and Rotated BRIEF)}. \\
Los hilos desempeñarán las siguientes funciones:
\begin{itemize}
    \item El \textit{tracking} se encargará de localizar la cámara en cada frame buscando marches entre las features
    del mapa local y minimizando la reprojección del error aplicando un \textit{Bundle Adjustment}\footnote{https://homes.cs.washington.edu/~sagarwal/bal.pdf} sólo de 
    movimiento.
    \item El \textit{Local Mapping} se encargará de gestionar y optimizar el mapa local aplicando un \textit{BA} local.
    \item El hilo de \textit{Loop Closing} se encargará de detectar bucles cerrados grandes y corregir la deriva acumulada
    realizando una optimización del grafo del \textit{pose} obtenido. Este hilo, lanza 4 hilos internos que se encagarán de
    realizar un \textit{Bundle Adjustement} completo tras la optimización del grafo del \textit{pose}, para hayar la solución
    más optima.
\end{itemize}

En caso de que el sistema pierda el tracking, lleva integrado un modulo de \textit{Plane recognition} basado en un vocabulario 
de palabras que básicamente es una especie de preentrenamiento de la técnica, para poder obtener y asociar keyframes más fácilmente. \\

\subsubsection{Análisis de resultados obtenidos}
En la imagen a continuación, puede observarse la técnica corriendo en un ordenador. Destacar que ésta técnica de SLAM presenta una carga computacional bastante elevada.
\begin{figure}[h!]
    \centering
    \includegraphics[width=1\textwidth]{images/working_zone_orb}
    \caption{ORB-SLAM2 en funcionamiento}
\end{figure}

Se observa en la imagen superior izquierda las features que está obteniendo en ese frame, a partir de las cuales se ha creado en mapa local y,
por extensión el global. En la imagen inferior izquierda, se puede observar el mapa creado, el cuál se analizará más tarde.\\
Por último, a la derecha se tienen las terminales a partir de las cuales se ha lanzado el SLAM y se realiza la comunicación con el robot.